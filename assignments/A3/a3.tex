\documentclass[11pt]{article}
\usepackage{amsmath}
\usepackage[utf8]{inputenc}
\usepackage[margin=0.75in]{geometry}

\title{CSC111 Assignment 3: Graphs, Recommender Systems, and Clustering}
\author{Azalea Gui \& Peter Lin}
\date{\today}

\newcommand{\N}{\mathbb{N}}
\newcommand{\Z}{\mathbb{Z}}
\newcommand{\R}{\mathbb{R}}
\newcommand{\cO}{\mathcal{O}}
\newcommand{\floor}[1]{\left\lfloor #1 \right\rfloor}
\newcommand{\code}[1]{\texttt{#1}}

\begin{document}
\maketitle

\section*{Part 1: The book review graph and simple recommendations}

\begin{enumerate}

\item[1.]
Complete this part in the provided \texttt{a3\_part1.py} starter file.
Do \textbf{not} include your solution in this file.

\item[2.]
Running Time Analysis for \texttt{load\_review\_graph}:

Let $n$ be the number of lines in \texttt{book\_names\_file}, let $m$ be the number of lines in \texttt{reviews\_file}.

There are two operations that involves iteration in the function, one reads the book names file and creates the \texttt{mp} dictionary, and the other one reads the reviews file and adds vertices to the graph.

In creating $mp$, the program first opened the file and created a \texttt{csv.reader}, which are both constant-time operations. Then, the dictionary comprehension statement loops through all $n$ lines, running only constant-time operations in each iteration for adding the book id and name pair into the dictionary, resulting in a running time of $\Theta(n)$. Summing up all the operations for creating $mp$ and ignoring constant-time operations, the running time would be $\in \Theta(n)$.

For adding the vertices, it also opened the file and created a \texttt{csv.reader} in constant time. Then, the loop iterates through all $m$ lines. In each iteration, two vertices and one edge are added, and it also accessed $mp$ to retrieve the book name, which are all constant time operations. Therefore, the total running time would be contained by $\in \Theta(m)$.

Since there are only constant-time operations outside the two iterating operations, the total running time of the function would be $\in \Theta(m + n)$

\item[3.]
Complete this part in the provided \texttt{a3\_part1.py} starter file.
Do \textbf{not} include your solution in this file.

\item[4.]
Complete this part in the provided \texttt{a3\_part1.py} starter file.
Do \textbf{not} include your solution in this file.

\end{enumerate}

\section*{Part 2: Weighted graphs, recommendations, review prediction}

Complete this part in the provided \texttt{a3\_part2\_recommendations.py} and \texttt{a3\_part2\_predictions.py} starter files.
Do \textbf{not} include your solution in this file.

\newpage

\section*{Part 3: Finding book clusters}

\begin{enumerate}

\item[1.]
Complete this part in the provided \texttt{a3\_part3.py} starter file.
Do \textbf{not} include your solution in this file.

\item[2.]
Complete this part in the provided \texttt{a3\_part3.py} starter file.
Do \textbf{not} include your solution in this file.

\item[3.]

\begin{enumerate}
\item[(a)]
TODO: Write your answer here.

\item[(b)]
TODO: Write your answer here.

\item[(c)]
TODO: Write your answer here.

\item[(d)]
\emph{Not to be handed in.}
\end{enumerate}

\end{enumerate}
\end{document}
